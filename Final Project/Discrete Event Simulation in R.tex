\documentclass[titlepage]{article}
\author{\Large \newline Joseph Shaw\, Matt Medina\, Dylon Dickinson}
\title{\Huge Discrete Event Simulation in R}

\begin{document}
	\maketitle
	
\clearpage


\vspace{-1in}

%Add Image
%\includegraphics[scale=.60]{image.jpg}

\tableofcontents

\Large

1 \hspace{0.1in} DES

2 \hspace{0.1in} Event-Oriented DES

3 \hspace{0.1in} Process-Oriented DES

4 \hspace{0.1in} Rposim

5 \hspace{0.1in} Simmer

6 \hspace{0.1in} Team member contributions

\hspace{0.25in} 6.1 \hspace{0.1in} Write-up

\hspace{0.25in} 6.2 \hspace{0.1in} Code

7 \hspace{0.1in} Appendix

\clearpage




\Large \textbf{What is DES?:} \normalsize
DES is discrete event simulation. The events that are being simulated are discontinous or discrete. We model real world situations of events on a smaller scale so that we can make a blueprint of what will happen in the actual model. There are three general approaches to doing discrete event simulation, which are Activity, Event, and Process Oriented DES. All three of these approaches use a queue to store either a list of the events happening or processes. They also all need to keep track of the environment time so that each event can be simulated in a certain amount of time. This time is also needed in order to tell when certain events are going to be happening, and for when a resource is available for something to use. We will look at two of them, Event-Oriented and Process Oriented DES. While they do the same thing, the two different methods of solving how to do DES are very different.

$\newline$

\Large \textbf{Event-Oriented DES:} \normalsize 
Event-oriented discrete event simulation is built around a scheduling of events.  Events are stored in an event list and executed in an ascending order of scheduled time.  Each event is given an event type so that each type can be processed according to the specifications of the event.  As with all discrete event simulation, use of a global time is necessary to keep a running track of variable times and number of event occurrences, so that statistical data may be gathered at the end of the simulation.

$\newline$

The DES package for R uses an event-oriented DES.  The events are stored in an event list attribute contained within a R environment named simlist.  The event list in the DES package is represented as a matrix that stores 1 event per row, with the first column being the event time and the second column being the event type.  The user has the capability of appending more user defined columns for other event purposes that may be useful for the discrete event simulation.  Each event type is signified by the use of an R numeric, which is crucial for the purposes of distinguishing between events so that they may be handled accordingly.

$\newline$

\Large \textbf{Process-Oriented DES:} \normalsize 
Process-Oriented descrete event simulation differs from Event-Oriented discrete event simulation in that it uses threading in order to allow multiple processes to interact at the same time. Processes can be in different states, such as running(or active), ready, idle, or terminated. The simulation will run until each process has terminated, with each process given a certain amount of time that it needs to run for. Like for Even-Oriented DES, we use an Queue to hold the events, but the events are generally tracked as time, and the processes will run until they receive a command to hold, idle, or terminate. An example of this comes from Simpy, the python package for DES.

$\newline$
With Simpy, we make a class that takes in the process class of simpy in order to be able to run the process. This makes it a subclass of the process class of simpy, which needs to be initialized in order to be run. This also requires us to make a run function for the simulation class to activate. This run function is in fact a generator, which is used to tell the OS when to allow the subprocesses to run, remain idle, or terminate, which typically returns a tuple with the amount of time. This generator allows for Simpy to be able to do multi-threaded processes during its simulation. Each instance of a class that has the run function within it will basically be a thread, while the os itself is another thread. The operations for yield are hold, release, passivate, and request. Hold will tell the simulation to stop running until the environment time his the hold time, request makes a process get in line to use a resource that is needed, release allows the next process to use a resource, and passivate will make the process wait until another process awakens it.
$\newline$
Simpy also uses resources in order for the threads to be able to interact with each other. 

$\newline$

\Large \textbf{The Rposim Package:} \normalsize 
$\newline$
\Large \textbf{4.1 Explanation of the package} \normalsize 
$\newline$
Our Rposim package attemps to be similar to Simpy, however without generators or threading the task is a bit difficult. We opted to use bigmemory to store our global variables into a matrix which allows us to create multiple instances of R and allows a shared data location in memory for each instance to access and write data to. Because of these issues, we require users of the package to use specific commands in their applications. They need to reference the bigmemory that is going to be used in the simulation so that the globals can be writted, otherwise we get race conditions that make the program run forever.
$\newline$
We also require users to use an S4 class for their application, so that we can retrieve the process id for the class. These process ids will be stored in bigmemory and will be used by the library in order to do certain yield the requests that the user may have. This is how we make up for the lack of generators in R. Each process id has a vector of values which tells the simulation whether it's running or not, whether theres a hold or not, and the time of the next event in the simulation. While we do have these restrictions, the rest of the code is general and up to the user to create. 
$\newline$
As stated before though, we tried making our syntax look as similar as possible to simpy so that there would be less issues for those that wanted to port their code to R from simpy. Our functions are in the form of yield_hold, yield_request, yield_release, yield_passivate, now, cancel, reactivate, initialize, activate, and simulate. Of which initialize, activate, and simulate are required in order to run any simulation. Their arguments are very close to what simpy has, with exception to activate, which takes the class instead of the class and a function. One of the race conditions that was showing up was coming from trying to insert the function, so we decided to just take the class. 

$\newline$
\Large \textbf{4.2 Differences across different  DES methods and TCP/IP method} \normalsize 
$\newline$
If we compare our pacakge to the DES.R package, there is a huge difference right off the bat in that the DES.R package is event-oriented DES while our package is process-oriented DES.
$\newline$
Comparing our package to Simpy, we notice that simpy can use generators while we cannot, we have to make our own methods that handle the yield commands that simpy has. As stated in the previous section, we modeled our package to be similar to Simpy. However, making a 1:1 port is impossible due to the restrictions of R. This led us to have to use bigmemory in order to solve the issue of having no generators and not being able to do parallel programming.
$\newline$
Now to compare the package that should be most similar to our Rposim package, Simmer. Simmer mainly uses an environment, trajectories, resources, and library made generators to run its simulation, while we rely soley on global variables and our own functions to simulate generators. Simmer uses a unique way of adding arrival events to try and simulate a generator to update the environment time. We just use Bigmemory to update our global variable of simulation time so that each simulation knows when to stop. Also as stated before, if event arrivals stop in simmer, the simulation stops, while in our package, we are automatically creating packages from the user defined function until the simulation time is done. 
$\newline$
We could use a TCP/IP approach in order to solve the issue of R having no threads, instead of shared-memory. This would involve using a server that would hold all the data and resources that the processes need and the processes would be clients that connect to the server. We could then have the server run the simulation for each process and easily receive codes for when to hold the simulation, and when to passivate and reactivate them. One of the processes would be the server that would send the server the operations that each of the other processes would need to be done on them. When the simulation is over, the os would then pull the data that is needed and send it to the application that is calling the library.
$\newline$

\Large \textbf{The Simmer Package:} \normalsize
TEXT HERE

$\newline$

\Large \textbf{Contributions:} \normalsize 
TEXT HERE

\hspace{0.25in} Write-ups:

\hspace{0.5in}	DES - Joseph Shaw

\hspace{0.5in}	Event-Oriented DES - Matt Medina

\hspace{0.5in}	Process-Oriented Des - Joseph Shaw

\hspace{0.5in}	The Rposim Package - Dylon Dickinson, Joseph Shaw

\hspace{0.5in}	The Simmer Package - Matt Medina

$\newline$

\hspace{0.25in} Code:

\hspace{0.5in}	1. Rposim - Dylon Dickinson, Joseph Shaw

\hspace{0.75in}	1.1 RposimMachRep1.r - Dylon Dickinson, Joseph Shaw

\hspace{0.75in}	1.2 RposimMachRep2.r - Dylon Dickinson, Joseph Shaw

\hspace{0.75in}	1.3 RposimMachRep3.r - Dylon Dickinson, Joseph Shaw

\hspace{0.5in}	2. SimmerMachRep1.r - Matt Medina, Joseph Shaw

$\newline$

\clearpage




\Large \textbf{Appendix:} \normalsize

\begin{verbatim}
Rposim.r

\end{verbatim}

\begin{verbatim}
RposimMachRep1.r

\end{verbatim}

\begin{verbatim}
#RposimMachRep2.r

	source("Rposim.R")

	#global variables
	UpRate = 1/1.0 # reciprocal of mean up time
	RepairRate = 1/0.5 # reciprocal of mean repair time
	NextID = 0 # next available ID number for MachineClass objects
	TotalUpTime = 0.0 # total up time for all machines
	# RepairPerson = resource(1)
	# NRep <<- 0
	# NImmedRep <<- 0

	MachineClass <- setRefClass("MachineClass",
	fields = list(StartUpTime="numeric",ID="numeric"),
	contains = "Process",
	methods = list(
	initialize = function()
	{
		.self$StartUpTime <<- 0.0
		.self$ID <<- NextID
		NextID <<- NextID + 1
		NUp <<- NUp + 1

	},
	Run = function()
	{
			while(1)
		{  
			# record current time, now(), so can see how long machine is up
			.self$StartUpTime <- now()
			# hold for exponentially distributed up time
			UpTime <- rexp(1,UpRate)
			yield_hold(.self, UpTime) # simulate UpTime
			TotalUpTime <<- TotalUpTime + now() - .self$StartUpTime
			# NRep = NRep + 1
			# if (RepairPerson$n == 1){
			#	NImmedRep = NImmedRep + 1
			#}
			#yield_request(RepairPerson)
			RepairTime <- rexp(1,RepairRate)
			# hold for exponentially distributed repair time
			yield_hold(.self, RepairTime)
		}
	}))
	main <- function()
	{
		  initialize()
		# set up the two machine threads
		for(i in 1:2)
		{
			# create a MachineClass object
			M <- MachineClass()
			activate(M,M$Run()) # required
		}
		# run until simulated time 10000
		MaxSimtime = 10000.0
		simulate(MaxSimtime) # required
			
		paste("the percentage of up time was ", TotalUpTime/(2*MaxSimtime))
		# paste("the percentage of times repair was immediate: ", NImmedRep/NRep)

	 }
	 
	 main()

\end{verbatim}

\begin{verbatim}
	#RposimMachRep3.r

	source("Rposim.R")

	#global variables
	UpRate = 1/1.0 # reciprocal of mean up time
	RepairRate = 1/0.5 # reciprocal of mean repair time
	NextID = 0 # next available ID number for MachineClass objects
	TotalUpTime = 0.0 # total up time for all machines
	# RepairPerson = resource(1)
	# NUp = 0
	# MachineList = c(0)

	MachineClass <- setRefClass("MachineClass",
	fields = list(StartUpTime="numeric",ID="numeric"),
	contains = "Process",
	methods = list(
	initialize = function()
	{
	    .self$StartUpTime <<- 0.0
	    .self$ID <<- NextID
	    NextID <<- NextID + 1
	    NUp <<- NUp + 1

	},
	Run = function()
	{
			while(1)
	    {  
	        # record current time, now(), so can see how long machine is up
	        .self$StartUpTime <- now()
	        # hold for exponentially distributed up time
	        UpTime <- rexp(1,UpRate)
	        yield_hold(.self, UpTime) # simulate UpTime
			TotalUpTime <<- TotalUpTime + now() - .self$StartUpTime
	        # NUp = NUp - 1
	        # if(NUp == 1){
	        #	yield_passivate(self)
	        # }else if(RepairPerson$n == 1){
	        #	reactivate(MachineList[1-self.ID])
	        #}
	        # yield_request(RepairPerson)
	        # 
	        RepairTime <- rexp(1,RepairRate)
	        # hold for exponentially distributed repair time
	        yield_hold(.self, RepairTime)
	        # NUp = NUp + 1
	        # yield_release(RepairPerson)
	    }
	}))
	main <- function()
	{
		  initialize()
	    # set up the two machine threads
	    for(i in 1:2)
	    {
	        # create a MachineClass object
	        M <- MachineClass()
	        activate(M,M$Run()) # required
	    }
	    # run until simulated time 10000
	    MaxSimtime = 10000.0
	    simulate(MaxSimtime) # required
			
	    paste("the percentage of up time was ", TotalUpTime/(2*MaxSimtime))
	    # paste("the percentage of times repair was immediate: ", NImmedRep/NRep)

	 }
	 
	 main()

\end{verbatim}

\clearpage

\begin{verbatim}
	#SimmerMachRep1.r

    library(simmer)

    NUM_MACHINES <- 2  # Number of machines
    NUM_REPAIRPERSONS <- 2 # Number of repair people
    REPAIRRATE <- 1/0.5    # Reciprocal of mean repair time
    UPRATE <- 1/1.0        # Reciprocal of mean up time 
    SIM_TIME <- 10000      # Simulation time 

    #setup
    set.seed(12345)
    env1 <- simmer()

    repair_person <- trajectory() %>%
        seize("fix", 1) %>%
        set_attribute("newUptime",function(){rexp(1,1/1.0)},global=TRUE) %>%
        timeout(function() {ifelse(is.na(get_attribute(env1,"newUptime")),0,
            get_attribute(env1,"newUptime"))}) %>%
        timeout(function(){rexp(1,REPAIRRATE)}) %>%
        release("fix", 1) 

    env1 %>%
        add_resource("fix", NUM_MACHINES) %>%
        add_generator("repair_person", repair_person, at(rep(0,NUM_REPAIRPERSONS))) %>%
        add_generator("create_new_uptime", repair_person, 
            function() sample(rexp(1,1/1.0), 1)) %>%

    # start the simulation
    run(until = SIM_TIME)

    x = env1 %>% get_mon_attributes
    y = x$value
    z = sum(y)
    PercentUp = z / (NUM_MACHINES*SIM_TIME)
    print(paste0("Percent machines are up is: ", PercentUp))

\end{verbatim}

\clearpage

\end{document}
